\section{开始实验}
\zhlipsum[2]\cite{HPC}

计数列表环境
\begin{enumerate}
	\item 这是一个计数列表环境.
	\item 这是一个计数列表环境.
	\item 这是一个计数列表环境.
\end{enumerate}

\vspace{2ex}
不计数列表环境
\begin{itemize}
	\item 这是一个不计数列表环境.
	\item 这是一个不计数列表环境.
	\item 这是一个不计数列表环境.
\end{itemize}



一些公式:

$$
		\left\{\begin{array}{l}
			\mbox{Find } u \in X \mbox{ such that } \\
			a(u, v)=F(v),~\forall v \in X.
		\end{array} \right. 
$$
has a unique solution. Moreover, we have
\begin{equation}
	 \|u\| \leq \frac{1}{\alpha}\|F\|_{X^{\prime}}. 
\end{equation}


表格模板:

\begin{table}
	\caption{这是一个三线表.}	
	\centering
	\begin{tabular}{ccc}
		\toprule
		\textbf{Treatments} & \textbf{Response 1} & \textbf{Response 2}\\
		\midrule
		Treatment 1 & 0.0003262 & 0.562 \\
		Treatment 2 & 0.0015681 & 0.910 \\
		\bottomrule
	\end{tabular}
\end{table}


本文定义了新的可变长度左中右 (LCR) 格式, LCR 三个格式会根据表格宽度的设定自行控制宽度, 且其宽度相等, 方便设置和页面相同宽度的表格. 本文还定义了 P\{\} 格式可以设定某一列宽度 (如 P\{1cm\} 控制某一列的宽度为 1cm) 并居中.%{0.9\textwidth}
\begin{table}[!htp]
	\centering
	% PLCR已经定义
	\caption{某校学生身高体重样本.}
	\label{tab2:heightweight}	
	\begin{tabular}{lccc}
		\toprule
		序号&年龄&身高&体重\\
		\midrule
		1&14&156&42\\
		2&16&158&45\\
		3&14&162&48\\
		4&15&163&50\\
		\cmidrule{2-4}
		平均&15&159.75&46.25\\
		\bottomrule
	\end{tabular}
\end{table}

表格示例
%\begin{table}[htp!]
%	\centering
%	\renewcommand\arraystretch{1.2} %定义表格高度
%	% PLCR前面已经定义
%	\caption{表格的描述.}
%	\label{tab3:NumError}
%	\begin{tabularx}{0.9\textwidth}{|P{1cm}|c|c|c|c|}
%		\Xhline{2\arrayrulewidth}
%		N  & A       & B    & C  & D   \\
%		\Xhline{2\arrayrulewidth}
%		1  & 9.20E-05 & 9.90E-05 & 1.00E-06 & 8.00E-06  \\
%		2  & 9.80E-05 & 8.00E-05 & 7.00E-06 & 1.40E-05  \\
%		3  & 4.00E-06 & 8.10E-05 & 8.80E-05 & 2.00E-05 \\
%		4  & 8.50E-05 & 8.70E-05 & 1.90E-05 & 2.10E-05 \\
%		5 & 8.60E-05 & 9.30E-05 & 2.50E-05 & 2.00E-06  \\
%		6 & 1.70E-05 & 2.40E-05 & 7.60E-05 & 8.30E-05  \\
%		7 & 2.30E-05 & 5.00E-06 & 8.20E-05 & 8.90E-05 \\
%		8 & 7.90E-05 & 6.00E-06 & 1.30E-05 & 9.50E-05  \\
%		9 & 1.00E-05 & 1.20E-05 & 9.40E-05 & 9.60E-05 \\
%		\Xhline{2\arrayrulewidth}
%	\end{tabularx}
%\end{table}


\begin{lstlisting}[caption = cs代码表测试]
	import keras
	from keras import layers
	
	def train_model(maxword, maxlen):
	model = keras.Sequential()
	
	# 前面数据需要的“词向量化”的操作,不算双向RNN的要求:
	model.add( layers.Embedding(maxword, 50, input_length=maxlen) )
	
	# 双向RNN搭建:
	model.add( layers.Bidirectional( layers.LSTM(64, dropout = 0.2, recurrent_dropout = 0.5) ) )
	
	# 外接一个单独的dropout层:非必须
	model.add( layers.Dropout(0.2) )
	
	# 进入全连接层:二分类,1个神经元就够
	model.add( layers.Dense(1, activation='sigmoid') )
	
	# 网络编译也在这个函数内完成:内容没变化
	model.compile( optimizer='adam',
	loss = 'binary_crossentropy',
	metrics = ['acc']
	) 
	return model
\end{lstlisting}