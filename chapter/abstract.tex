\begin{abstract}
    该作业报告完成了本学期基因组学课程布置的三道题目.
    
    作业第1题是“介绍Oxford Nanopore的原理、优缺点、应用”.对于原理,我主要介绍了该公司推出的首款纳米孔测序设备MinION\cite{2jain2016oxford}的技术原理,以及两种常见的Basecaller电流信号解码算法Nanocall\cite{david2017nanocall}和DeepNano\cite{bovza2017deepnano};
    对于优缺点,我讨论了长读长、高通量、实时靶向测序、直接检测碱基修饰等6种优点,以及准确性较低、高错误率等3种缺点,并与第一、二代测序进行比较;
    对于应用,我主要讨论了填补参考基因组中的空缺、建立新的参考基因组、识别大型结构变异等6种可能的应用场景.
    
    作业第2题是“简述我知道的RNA的种类和功能”.我简要概述了9种RNA的种类和功能.包括:mRNA(信使RNA)在基因表达过程中负责转导遗传信息;tRNA(转运RNA)在蛋白质合成过程中传递氨基酸;rRNA(核糖体RNA)作为核糖体的主要成分,参与蛋白质合成;snRNA(小核RNA)和snoRNA(小核仁RNA)参与RNA剪接和修饰;miRNA(微小RNA)调控基因表达,影响生物进程;lncRNA(长非编码RNA)在转录调控、基因沉默等方面发挥作用;piRNA(Piwi-interacting RNA)参与生殖细胞中的转座子沉默;circRNA(环状RNA)作为miRNA的海绵分子,调控基因表达.
    
    作业第3题是“泛基因组学(Pan-genome):内容、应用场景、研究实例”.对于内容,我主要介绍了泛基因组学的定义以及常见概念;对于应用场景,我给出作物基因组学、育种和进化研究、研究不同品种结构变异影响基因差异表达和结合GWAS数据捕获更完整的遗传变异信息共3种应用场景;对于研究实例,我选取了2023年3月在《Nature Communications》上发表的“亚洲水稻泛基因组倒位指数”文章《Pan-genome inversion index reveals evolutionary insights into the subpopulation structure of Asian rice》\cite{zhou2023pan},从研究背景、倒位指数、系统发育树、鉴定与评估等多个方面展开讨论,最后认为泛基因组学在大规模组学数据和科学计算时代可以发挥非常重要的作用.
    
    该作业使用LaTeX排版,编译所需的全部文件已经上传到\url{https://github.com/StatXzy7/lzu_genomics_hw}.
    
\end{abstract}

\textbf{关键词:} 基因组学; Oxford Nanopore; RNA; 泛基因组学
\newpage